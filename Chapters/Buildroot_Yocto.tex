\section{De Buildroot à Yocto}

\subsection{Buildroot}
\paragraph{}
Buildroot est un outil qui simplifie et automatise le processus de construction d'un système Linux complet pour un système embarqué, 
en utilisant la compilation croisée.

Pour y parvenir, Buildroot est capable de générer une toolchain, un RFS (root file system),
 un kernel Linux et un bootloader pour la cible. Buildroot peut être utilisé pour toute combinaison de ces options, 
 indépendamment (vous pouvez par exemple utiliser une chaîne de compilation croisée existante, 
 et ne construire que votre système de fichiers racine avec Buildroot). Ce qui permet de simplifier le processus en laissant tout de même beaucoup de liberté.

Les systèmes embarqués utilisent souvent des processeurs qui ne sont pas les processeurs x86 habituels que tout le monde est habitué à avoir dans son PC.
Créer son OS pour une architecture que peu de gens utilise s'avére très fastidieu.Il peut s'agir de processeurs PowerPC, de processeurs MIPS, de processeurs ARM, etc.
Buildroot rend le dévellopement del'OS adapté à ces processeurs très accessible.

\subsection{Yocto}
\paragraph{}
Yocto intègre des parties développées conjointement pour OpenEmbedded, notamment BitBake, OpenEmbedded-Core et d'autres métadonnées. 
Les éléments développés dans le cadre du projet, appelés "meta-yocto" et "meta-yocto-bsp", comprennent l'intégration des éclipses. 
Ensemble, ils améliorent les outils d'OpenEmbedded, cette plateforme de référence pour la construction de systèmes avancés embarqués dans les HW est connue 
sous le nom de Poky.

Pour développer des logiciels, nous avons besoin d'une chaîne d'outils (croisés) : les fichiers sources et les instructions sur la façon de les compiler. 
C'est suffisant pour une source. 
Pour plus de composants et de dépendances dans la compilation et le temps d'exécution, il faut augmenter la complexité et des étapes supplémentaires. 
Bitbake est un agent ayant la capacité d'interpréter et d'exécuter les recettes d'amélioration, il calcule la chaîne des tâches nécessaires pour développer l'objectif défini et exécuté. 

