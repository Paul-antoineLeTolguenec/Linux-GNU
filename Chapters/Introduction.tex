\section{Introduction}
\paragraph{}
Dans le monde de l'embarqué, il est souvent utile de créer son propre système d'exploitation. En effet, les contraintes de coûts et de consommation sont importante, amenant à travailler avec des processeurs et des mémoires très limités. Ces limitations rendent necessaire une grande optimisation de la partie logiciel, pour n'avoir que le stricte minimum. Pour cela, une séléction des composants de notre OS est nécessaire.
Or la construction d'un OS peut être longue et délicate. C'est pourquoi différents framework sont nées pour faciliter cette construction d'OS personnalisé. Un des plus connu et des plus facile d'utilisation est Buildroot. Mais Yocto, un projet récent, est en train de s'imposer comme la référence. Yocto se veut bien plus modulaire que Buildroot, en proposant une vaste gamme de configurations. Nous nous proposons dans ce rapport de donner les éléments de décisions pour un choix pertinant entre les deux framework.


\textbf{Repo GitHub}

\url{https://github.com/Paul-antoineLeTolguenec/Linux-GNU.git}

\newpage
