\section{Introduction}
\paragraph{}
Dans le monde de l'embarqué, il est souvent utile de créer son propre système d'exploitation. En effet, les contraintes de coûts et de consommation sont importantes, amenant à travailler avec des processeurs et des mémoires très limités. Ces limitations rendent nécessaires une grande optimisation de la partie logiciel, pour n'avoir que le strict minimum. Pour cela, une sélection des composants de notre OS est nécessaire.
Or la construction d'un OS peut être longue et délicate. C'est pourquoi différents frameworks sont nés pour faciliter cette construction d'OS personnalisée. Un des plus connus et des plus faciles d'utilisation est Buildroot. Mais Yocto, un projet récent plus récent et supporté par les grands acteurs de l'industrie tel Intel, est en train de s'imposer comme la référence. Yocto se veut bien plus modulaire que Buildroot, en proposant une vaste gamme de configurations, une gestion des paquets. Nous nous proposons dans ce rapport de donner les éléments de décisions pour un choix pertinent entre les deux frameworks.


\textbf{Repo GitHub}

\url{https://github.com/Paul-antoineLeTolguenec/Linux-GNU.git}

\newpage
