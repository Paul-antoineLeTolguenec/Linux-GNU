\section{Introduction}
\paragraph{}
Dans le monde de l'embarqué, il est souvent utile de créer son propre système d'exploitation. En effet, les contraintes imposées par le système sont diverses et variées en fonction du projet.
La construction d'un OS peut être longue et délicate. C'est pourquoi différents framework sont nées pour facilité cette construction d'OS personnalisé. Celui auqel nous allons nous intéresser est le projet Yocto.
La création d'un système d'exploitation personnalisé sur Linux conçu pour les spécificités d'une carte embarquée et les exigences du futur produit peut s'avérer difficile, mais le projet Yocto a pour mission d'aider. 
Le projet Yocto est un projet collaboratif en exploitation libre géré par la Fondation Linux. Mais la question qu'il convient de se poser est que dois-je utiliser pour l'OS de mon système.
Dans cet article, nous allons tenter d'évaluer les avantages et inconvénients des différents projets en fonction des différents systèmes.
Puis nous allons synthétiser le déroulement de la construction de l'OS pour les deux projets.
\textbf{Repo GitHub}

\url{https://github.com/Paul-antoineLeTolguenec/Linux-GNU.git}

\newpage
